\documentclass{article}

% Language setting
% Replace `english' with e.g. `spanish' to change the document language
\usepackage[english]{babel}

% Set page size and margins
% Replace `letterpaper' with `a4paper' for UK/EU standard size
\usepackage[letterpaper,top=2cm,bottom=2cm,left=3cm,right=3cm,marginparwidth=1.75cm]{geometry}

% Useful packages
\usepackage{amsmath}
\usepackage{amsfonts} 
\usepackage{graphicx}
\usepackage[colorlinks=true, allcolors=blue]{hyperref}

\title{TDT4171 — Artificial Intelligence Methods \\ Assignment 4 - Making simple decisions}
\author{Erik Storås Sommer - 535006}
\date{February 2023}

\begin{document}
\maketitle

\section*{Exercise 1}

\subsection*{a.}

The first Bayes net (i) can not correctly represent P(Flavor,Wrapper,Shape) because Wrapper and Shape are dependant of eachother.

The second Bayes net (ii) correctly represents the dependencies between the variables, but it creates a cyclic graph, which is not desirable in a Bayesian network.

The third Bayes net (iii) represents the direct causal relationships between Wrapper, Flavor, and Shape, and it does not create a cyclic graph.

\subsection*{b.}

The best representation for this problem is the Bayes net (iii), since it does not create a cyclic graph and has the least amount of dependencies. Making this network the most compact representation that captures the causal relationships between the variables.


\subsection*{c.}

Yes, network (i) asserts that Wrapper is independent of Shape. The directed edge between Wrapper and Flavor in this network represents a causal relationship, meaning that Wrapper is dependent on Flavor but not on Shape. The absence of an edge between Wrapper and Shape in this network represents that Wrapper and Shape are conditionally independent given Flavor.

\subsection*{d.}

The probability that the flavor of the candy is strawberry, given that it is a round candy with a red wrapper, can be calculated as follows:

\[P(W = r) = P(W = r | F = s) \cdot P(F = s) + P(W = r | F = a) \cdot P(F = a)\]

where W = r is the event that the wrapper is red, F = s is the event that the flavor is strawberry, and F = a is the event that the flavor is anchovy.

Applying the probabilities given in the story, we get:

\[P(W = r) = 0.7 \cdot 0.8 + 0.3 \cdot 0.1 = 0.59\]

The the probability that the candy has a red wrapper is 59 percent.


\subsection*{e.}

Defining S = r as the event that the shape is round, and W = r as the event that the wrapper is red. We can then calculate the probability that the flavor is strawberry, denoted with F = s, given that the shape is round and the wrapper is red as follows:

\[P(F = s | S = r \wedge W = r) = \alpha P(S = r \wedge W = r | F = s) \cdot P(F = s)\]
\[= \alpha P(S = r | F = s) \cdot P(W = r | F = s) \cdot P(F = s)\]
\[= \alpha \langle 0.8 \cdot 0.8 \cdot 0.7, 0.1 \cdot 0.1 \cdot 0.3 \rangle \]
\[= \alpha \langle 0.448, 0.003 \rangle\]

We get:

\[P(F = s | S = r \wedge W = r) = \dfrac{0.448}{0.448 + 0.003} = 0.993\]

We see that the probability that the candy flavor is strawberry is \(>\) 99 percent higher than the probability that the flavor is anchovy.

\subsection*{f.}

Given the probabilities given in the story, the expression for the expected value of an unopened candy box is:

\[EV(Box) = 0.7s + 0.3a\]


\section*{Exercise 2}

\subsection*{a.}

To determine which option Mary would choose, we need to calculate the expected utility (EU) for each option and compare them.

The expected utility of receiving \$500 with certainty is:

\[EU1 = U(\$500) = -e^{-500/500} = -e^{-1} = -0.37\]

The expected utility of participating in the lottery is:

\[EU2 = 0.6 * U(\$5000) + 0.4 * U(0)\]
\[= 0.6 * -e^{-5000/500} + 0.4 * -e^{0/500}\]
\[= 0.6 * -e^{-10} + 0.4 * -1 = -0.4\]

Since EU1 is greater than EU2, Mary would choose to receive \$500 with certainty.

\subsection*{b.}

To find the value of R, we need to find the expected utility (EU) for each option and set them equal. The expected utility of receiving \$100 with certainty is:

\[EU1 = U(\$100) = -e^{-100/R}\]

The expected utility of participating in the lottery is:

\[EU2 = 0.5 * U(\$500) + 0.5 * U(0)\]
\[= 0.5 * -e^{-500/R} + 0.5 * -e^{0/R}\] 
\[= 0.5 * -e^{-500/R} + 0.5 * -1\]

Setting EU1 = EU2, we get:

\[-e^{-100/R} = 0.5 * -e^{-500/R} -0.5\]

\[-e^{-100/R} = -\dfrac{1}{2e^{500/R}} -0.5\]

\[-e^{-100/R} * 2e^{500/R} = (-\dfrac{1}{2e^{500/R}} -0.5) * 2e^{500/R}\]

\[-2e^{-400/R} = -\dfrac{1}{2e^{500/R}} -0.5 * 2e^{500/R}\]

\[-2e^{-400/R} = -1 - e^{500/R}\]

\[-2e^{-400/R} - (-1 - e^{500/R}) = 0\]

Solve the equation using substitution \(t=e^{100/R}\):

\[-2t^{4} + 1 + t^{5}\]

\[(1 - t)(t^3-t^4+t^2+t+1) = 0\]

Separate the equation into 2 possible cases

\[1 - t = 0\]
\[t^3-t^4+t^2+t+1 = 0\]

\[t = 1\]
\[t^3-t^4+t^2+t+1 = 0\]

\[t = 1\]
\[t \approx -0.7748\]
\[t \approx 1.927562\]

Substitute back into the original equation:

\[e^{100/R} = 1\]
\[e^{100/R} \approx -0.7748\]
\[e^{100/R} \approx 1.927562\]

\[R \in \emptyset\]
\[R \notin \mathbb{R}\]
\[e^{100/R} \approx 1.927562\]

Solve the equation for R

\[R = \dfrac{100}{ln(1.927562)}\]

\[R \approx 152.38\]

\end{document}